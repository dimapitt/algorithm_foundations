
% Default to the notebook output style

    


% Inherit from the specified cell style.




    
\documentclass[11pt]{article}

    
    
    \usepackage[T1]{fontenc}
    % Nicer default font (+ math font) than Computer Modern for most use cases
    \usepackage{mathpazo}

    % Basic figure setup, for now with no caption control since it's done
    % automatically by Pandoc (which extracts ![](path) syntax from Markdown).
    \usepackage{graphicx}
    % We will generate all images so they have a width \maxwidth. This means
    % that they will get their normal width if they fit onto the page, but
    % are scaled down if they would overflow the margins.
    \makeatletter
    \def\maxwidth{\ifdim\Gin@nat@width>\linewidth\linewidth
    \else\Gin@nat@width\fi}
    \makeatother
    \let\Oldincludegraphics\includegraphics
    % Set max figure width to be 80% of text width, for now hardcoded.
    \renewcommand{\includegraphics}[1]{\Oldincludegraphics[width=.8\maxwidth]{#1}}
    % Ensure that by default, figures have no caption (until we provide a
    % proper Figure object with a Caption API and a way to capture that
    % in the conversion process - todo).
    \usepackage{caption}
    \DeclareCaptionLabelFormat{nolabel}{}
    \captionsetup{labelformat=nolabel}

    \usepackage{adjustbox} % Used to constrain images to a maximum size 
    \usepackage{xcolor} % Allow colors to be defined
    \usepackage{enumerate} % Needed for markdown enumerations to work
    \usepackage{geometry} % Used to adjust the document margins
    \usepackage{amsmath} % Equations
    \usepackage{amssymb} % Equations
    \usepackage{textcomp} % defines textquotesingle
    % Hack from http://tex.stackexchange.com/a/47451/13684:
    \AtBeginDocument{%
        \def\PYZsq{\textquotesingle}% Upright quotes in Pygmentized code
    }
    \usepackage{upquote} % Upright quotes for verbatim code
    \usepackage{eurosym} % defines \euro
    \usepackage[mathletters]{ucs} % Extended unicode (utf-8) support
    \usepackage[utf8x]{inputenc} % Allow utf-8 characters in the tex document
    \usepackage{fancyvrb} % verbatim replacement that allows latex
    \usepackage{grffile} % extends the file name processing of package graphics 
                         % to support a larger range 
    % The hyperref package gives us a pdf with properly built
    % internal navigation ('pdf bookmarks' for the table of contents,
    % internal cross-reference links, web links for URLs, etc.)
    \usepackage{hyperref}
    \usepackage{longtable} % longtable support required by pandoc >1.10
    \usepackage{booktabs}  % table support for pandoc > 1.12.2
    \usepackage[inline]{enumitem} % IRkernel/repr support (it uses the enumerate* environment)
    \usepackage[normalem]{ulem} % ulem is needed to support strikethroughs (\sout)
                                % normalem makes italics be italics, not underlines
    

    
    
    % Colors for the hyperref package
    \definecolor{urlcolor}{rgb}{0,.145,.698}
    \definecolor{linkcolor}{rgb}{.71,0.21,0.01}
    \definecolor{citecolor}{rgb}{.12,.54,.11}

    % ANSI colors
    \definecolor{ansi-black}{HTML}{3E424D}
    \definecolor{ansi-black-intense}{HTML}{282C36}
    \definecolor{ansi-red}{HTML}{E75C58}
    \definecolor{ansi-red-intense}{HTML}{B22B31}
    \definecolor{ansi-green}{HTML}{00A250}
    \definecolor{ansi-green-intense}{HTML}{007427}
    \definecolor{ansi-yellow}{HTML}{DDB62B}
    \definecolor{ansi-yellow-intense}{HTML}{B27D12}
    \definecolor{ansi-blue}{HTML}{208FFB}
    \definecolor{ansi-blue-intense}{HTML}{0065CA}
    \definecolor{ansi-magenta}{HTML}{D160C4}
    \definecolor{ansi-magenta-intense}{HTML}{A03196}
    \definecolor{ansi-cyan}{HTML}{60C6C8}
    \definecolor{ansi-cyan-intense}{HTML}{258F8F}
    \definecolor{ansi-white}{HTML}{C5C1B4}
    \definecolor{ansi-white-intense}{HTML}{A1A6B2}

    % commands and environments needed by pandoc snippets
    % extracted from the output of `pandoc -s`
    \providecommand{\tightlist}{%
      \setlength{\itemsep}{0pt}\setlength{\parskip}{0pt}}
    \DefineVerbatimEnvironment{Highlighting}{Verbatim}{commandchars=\\\{\}}
    % Add ',fontsize=\small' for more characters per line
    \newenvironment{Shaded}{}{}
    \newcommand{\KeywordTok}[1]{\textcolor[rgb]{0.00,0.44,0.13}{\textbf{{#1}}}}
    \newcommand{\DataTypeTok}[1]{\textcolor[rgb]{0.56,0.13,0.00}{{#1}}}
    \newcommand{\DecValTok}[1]{\textcolor[rgb]{0.25,0.63,0.44}{{#1}}}
    \newcommand{\BaseNTok}[1]{\textcolor[rgb]{0.25,0.63,0.44}{{#1}}}
    \newcommand{\FloatTok}[1]{\textcolor[rgb]{0.25,0.63,0.44}{{#1}}}
    \newcommand{\CharTok}[1]{\textcolor[rgb]{0.25,0.44,0.63}{{#1}}}
    \newcommand{\StringTok}[1]{\textcolor[rgb]{0.25,0.44,0.63}{{#1}}}
    \newcommand{\CommentTok}[1]{\textcolor[rgb]{0.38,0.63,0.69}{\textit{{#1}}}}
    \newcommand{\OtherTok}[1]{\textcolor[rgb]{0.00,0.44,0.13}{{#1}}}
    \newcommand{\AlertTok}[1]{\textcolor[rgb]{1.00,0.00,0.00}{\textbf{{#1}}}}
    \newcommand{\FunctionTok}[1]{\textcolor[rgb]{0.02,0.16,0.49}{{#1}}}
    \newcommand{\RegionMarkerTok}[1]{{#1}}
    \newcommand{\ErrorTok}[1]{\textcolor[rgb]{1.00,0.00,0.00}{\textbf{{#1}}}}
    \newcommand{\NormalTok}[1]{{#1}}
    
    % Additional commands for more recent versions of Pandoc
    \newcommand{\ConstantTok}[1]{\textcolor[rgb]{0.53,0.00,0.00}{{#1}}}
    \newcommand{\SpecialCharTok}[1]{\textcolor[rgb]{0.25,0.44,0.63}{{#1}}}
    \newcommand{\VerbatimStringTok}[1]{\textcolor[rgb]{0.25,0.44,0.63}{{#1}}}
    \newcommand{\SpecialStringTok}[1]{\textcolor[rgb]{0.73,0.40,0.53}{{#1}}}
    \newcommand{\ImportTok}[1]{{#1}}
    \newcommand{\DocumentationTok}[1]{\textcolor[rgb]{0.73,0.13,0.13}{\textit{{#1}}}}
    \newcommand{\AnnotationTok}[1]{\textcolor[rgb]{0.38,0.63,0.69}{\textbf{\textit{{#1}}}}}
    \newcommand{\CommentVarTok}[1]{\textcolor[rgb]{0.38,0.63,0.69}{\textbf{\textit{{#1}}}}}
    \newcommand{\VariableTok}[1]{\textcolor[rgb]{0.10,0.09,0.49}{{#1}}}
    \newcommand{\ControlFlowTok}[1]{\textcolor[rgb]{0.00,0.44,0.13}{\textbf{{#1}}}}
    \newcommand{\OperatorTok}[1]{\textcolor[rgb]{0.40,0.40,0.40}{{#1}}}
    \newcommand{\BuiltInTok}[1]{{#1}}
    \newcommand{\ExtensionTok}[1]{{#1}}
    \newcommand{\PreprocessorTok}[1]{\textcolor[rgb]{0.74,0.48,0.00}{{#1}}}
    \newcommand{\AttributeTok}[1]{\textcolor[rgb]{0.49,0.56,0.16}{{#1}}}
    \newcommand{\InformationTok}[1]{\textcolor[rgb]{0.38,0.63,0.69}{\textbf{\textit{{#1}}}}}
    \newcommand{\WarningTok}[1]{\textcolor[rgb]{0.38,0.63,0.69}{\textbf{\textit{{#1}}}}}
    
    
    % Define a nice break command that doesn't care if a line doesn't already
    % exist.
    \def\br{\hspace*{\fill} \\* }
    % Math Jax compatability definitions
    \def\gt{>}
    \def\lt{<}
    % Document parameters
    \title{Programming Assignment 1}
    
    
    

    % Pygments definitions
    
\makeatletter
\def\PY@reset{\let\PY@it=\relax \let\PY@bf=\relax%
    \let\PY@ul=\relax \let\PY@tc=\relax%
    \let\PY@bc=\relax \let\PY@ff=\relax}
\def\PY@tok#1{\csname PY@tok@#1\endcsname}
\def\PY@toks#1+{\ifx\relax#1\empty\else%
    \PY@tok{#1}\expandafter\PY@toks\fi}
\def\PY@do#1{\PY@bc{\PY@tc{\PY@ul{%
    \PY@it{\PY@bf{\PY@ff{#1}}}}}}}
\def\PY#1#2{\PY@reset\PY@toks#1+\relax+\PY@do{#2}}

\expandafter\def\csname PY@tok@w\endcsname{\def\PY@tc##1{\textcolor[rgb]{0.73,0.73,0.73}{##1}}}
\expandafter\def\csname PY@tok@c\endcsname{\let\PY@it=\textit\def\PY@tc##1{\textcolor[rgb]{0.25,0.50,0.50}{##1}}}
\expandafter\def\csname PY@tok@cp\endcsname{\def\PY@tc##1{\textcolor[rgb]{0.74,0.48,0.00}{##1}}}
\expandafter\def\csname PY@tok@k\endcsname{\let\PY@bf=\textbf\def\PY@tc##1{\textcolor[rgb]{0.00,0.50,0.00}{##1}}}
\expandafter\def\csname PY@tok@kp\endcsname{\def\PY@tc##1{\textcolor[rgb]{0.00,0.50,0.00}{##1}}}
\expandafter\def\csname PY@tok@kt\endcsname{\def\PY@tc##1{\textcolor[rgb]{0.69,0.00,0.25}{##1}}}
\expandafter\def\csname PY@tok@o\endcsname{\def\PY@tc##1{\textcolor[rgb]{0.40,0.40,0.40}{##1}}}
\expandafter\def\csname PY@tok@ow\endcsname{\let\PY@bf=\textbf\def\PY@tc##1{\textcolor[rgb]{0.67,0.13,1.00}{##1}}}
\expandafter\def\csname PY@tok@nb\endcsname{\def\PY@tc##1{\textcolor[rgb]{0.00,0.50,0.00}{##1}}}
\expandafter\def\csname PY@tok@nf\endcsname{\def\PY@tc##1{\textcolor[rgb]{0.00,0.00,1.00}{##1}}}
\expandafter\def\csname PY@tok@nc\endcsname{\let\PY@bf=\textbf\def\PY@tc##1{\textcolor[rgb]{0.00,0.00,1.00}{##1}}}
\expandafter\def\csname PY@tok@nn\endcsname{\let\PY@bf=\textbf\def\PY@tc##1{\textcolor[rgb]{0.00,0.00,1.00}{##1}}}
\expandafter\def\csname PY@tok@ne\endcsname{\let\PY@bf=\textbf\def\PY@tc##1{\textcolor[rgb]{0.82,0.25,0.23}{##1}}}
\expandafter\def\csname PY@tok@nv\endcsname{\def\PY@tc##1{\textcolor[rgb]{0.10,0.09,0.49}{##1}}}
\expandafter\def\csname PY@tok@no\endcsname{\def\PY@tc##1{\textcolor[rgb]{0.53,0.00,0.00}{##1}}}
\expandafter\def\csname PY@tok@nl\endcsname{\def\PY@tc##1{\textcolor[rgb]{0.63,0.63,0.00}{##1}}}
\expandafter\def\csname PY@tok@ni\endcsname{\let\PY@bf=\textbf\def\PY@tc##1{\textcolor[rgb]{0.60,0.60,0.60}{##1}}}
\expandafter\def\csname PY@tok@na\endcsname{\def\PY@tc##1{\textcolor[rgb]{0.49,0.56,0.16}{##1}}}
\expandafter\def\csname PY@tok@nt\endcsname{\let\PY@bf=\textbf\def\PY@tc##1{\textcolor[rgb]{0.00,0.50,0.00}{##1}}}
\expandafter\def\csname PY@tok@nd\endcsname{\def\PY@tc##1{\textcolor[rgb]{0.67,0.13,1.00}{##1}}}
\expandafter\def\csname PY@tok@s\endcsname{\def\PY@tc##1{\textcolor[rgb]{0.73,0.13,0.13}{##1}}}
\expandafter\def\csname PY@tok@sd\endcsname{\let\PY@it=\textit\def\PY@tc##1{\textcolor[rgb]{0.73,0.13,0.13}{##1}}}
\expandafter\def\csname PY@tok@si\endcsname{\let\PY@bf=\textbf\def\PY@tc##1{\textcolor[rgb]{0.73,0.40,0.53}{##1}}}
\expandafter\def\csname PY@tok@se\endcsname{\let\PY@bf=\textbf\def\PY@tc##1{\textcolor[rgb]{0.73,0.40,0.13}{##1}}}
\expandafter\def\csname PY@tok@sr\endcsname{\def\PY@tc##1{\textcolor[rgb]{0.73,0.40,0.53}{##1}}}
\expandafter\def\csname PY@tok@ss\endcsname{\def\PY@tc##1{\textcolor[rgb]{0.10,0.09,0.49}{##1}}}
\expandafter\def\csname PY@tok@sx\endcsname{\def\PY@tc##1{\textcolor[rgb]{0.00,0.50,0.00}{##1}}}
\expandafter\def\csname PY@tok@m\endcsname{\def\PY@tc##1{\textcolor[rgb]{0.40,0.40,0.40}{##1}}}
\expandafter\def\csname PY@tok@gh\endcsname{\let\PY@bf=\textbf\def\PY@tc##1{\textcolor[rgb]{0.00,0.00,0.50}{##1}}}
\expandafter\def\csname PY@tok@gu\endcsname{\let\PY@bf=\textbf\def\PY@tc##1{\textcolor[rgb]{0.50,0.00,0.50}{##1}}}
\expandafter\def\csname PY@tok@gd\endcsname{\def\PY@tc##1{\textcolor[rgb]{0.63,0.00,0.00}{##1}}}
\expandafter\def\csname PY@tok@gi\endcsname{\def\PY@tc##1{\textcolor[rgb]{0.00,0.63,0.00}{##1}}}
\expandafter\def\csname PY@tok@gr\endcsname{\def\PY@tc##1{\textcolor[rgb]{1.00,0.00,0.00}{##1}}}
\expandafter\def\csname PY@tok@ge\endcsname{\let\PY@it=\textit}
\expandafter\def\csname PY@tok@gs\endcsname{\let\PY@bf=\textbf}
\expandafter\def\csname PY@tok@gp\endcsname{\let\PY@bf=\textbf\def\PY@tc##1{\textcolor[rgb]{0.00,0.00,0.50}{##1}}}
\expandafter\def\csname PY@tok@go\endcsname{\def\PY@tc##1{\textcolor[rgb]{0.53,0.53,0.53}{##1}}}
\expandafter\def\csname PY@tok@gt\endcsname{\def\PY@tc##1{\textcolor[rgb]{0.00,0.27,0.87}{##1}}}
\expandafter\def\csname PY@tok@err\endcsname{\def\PY@bc##1{\setlength{\fboxsep}{0pt}\fcolorbox[rgb]{1.00,0.00,0.00}{1,1,1}{\strut ##1}}}
\expandafter\def\csname PY@tok@kc\endcsname{\let\PY@bf=\textbf\def\PY@tc##1{\textcolor[rgb]{0.00,0.50,0.00}{##1}}}
\expandafter\def\csname PY@tok@kd\endcsname{\let\PY@bf=\textbf\def\PY@tc##1{\textcolor[rgb]{0.00,0.50,0.00}{##1}}}
\expandafter\def\csname PY@tok@kn\endcsname{\let\PY@bf=\textbf\def\PY@tc##1{\textcolor[rgb]{0.00,0.50,0.00}{##1}}}
\expandafter\def\csname PY@tok@kr\endcsname{\let\PY@bf=\textbf\def\PY@tc##1{\textcolor[rgb]{0.00,0.50,0.00}{##1}}}
\expandafter\def\csname PY@tok@bp\endcsname{\def\PY@tc##1{\textcolor[rgb]{0.00,0.50,0.00}{##1}}}
\expandafter\def\csname PY@tok@fm\endcsname{\def\PY@tc##1{\textcolor[rgb]{0.00,0.00,1.00}{##1}}}
\expandafter\def\csname PY@tok@vc\endcsname{\def\PY@tc##1{\textcolor[rgb]{0.10,0.09,0.49}{##1}}}
\expandafter\def\csname PY@tok@vg\endcsname{\def\PY@tc##1{\textcolor[rgb]{0.10,0.09,0.49}{##1}}}
\expandafter\def\csname PY@tok@vi\endcsname{\def\PY@tc##1{\textcolor[rgb]{0.10,0.09,0.49}{##1}}}
\expandafter\def\csname PY@tok@vm\endcsname{\def\PY@tc##1{\textcolor[rgb]{0.10,0.09,0.49}{##1}}}
\expandafter\def\csname PY@tok@sa\endcsname{\def\PY@tc##1{\textcolor[rgb]{0.73,0.13,0.13}{##1}}}
\expandafter\def\csname PY@tok@sb\endcsname{\def\PY@tc##1{\textcolor[rgb]{0.73,0.13,0.13}{##1}}}
\expandafter\def\csname PY@tok@sc\endcsname{\def\PY@tc##1{\textcolor[rgb]{0.73,0.13,0.13}{##1}}}
\expandafter\def\csname PY@tok@dl\endcsname{\def\PY@tc##1{\textcolor[rgb]{0.73,0.13,0.13}{##1}}}
\expandafter\def\csname PY@tok@s2\endcsname{\def\PY@tc##1{\textcolor[rgb]{0.73,0.13,0.13}{##1}}}
\expandafter\def\csname PY@tok@sh\endcsname{\def\PY@tc##1{\textcolor[rgb]{0.73,0.13,0.13}{##1}}}
\expandafter\def\csname PY@tok@s1\endcsname{\def\PY@tc##1{\textcolor[rgb]{0.73,0.13,0.13}{##1}}}
\expandafter\def\csname PY@tok@mb\endcsname{\def\PY@tc##1{\textcolor[rgb]{0.40,0.40,0.40}{##1}}}
\expandafter\def\csname PY@tok@mf\endcsname{\def\PY@tc##1{\textcolor[rgb]{0.40,0.40,0.40}{##1}}}
\expandafter\def\csname PY@tok@mh\endcsname{\def\PY@tc##1{\textcolor[rgb]{0.40,0.40,0.40}{##1}}}
\expandafter\def\csname PY@tok@mi\endcsname{\def\PY@tc##1{\textcolor[rgb]{0.40,0.40,0.40}{##1}}}
\expandafter\def\csname PY@tok@il\endcsname{\def\PY@tc##1{\textcolor[rgb]{0.40,0.40,0.40}{##1}}}
\expandafter\def\csname PY@tok@mo\endcsname{\def\PY@tc##1{\textcolor[rgb]{0.40,0.40,0.40}{##1}}}
\expandafter\def\csname PY@tok@ch\endcsname{\let\PY@it=\textit\def\PY@tc##1{\textcolor[rgb]{0.25,0.50,0.50}{##1}}}
\expandafter\def\csname PY@tok@cm\endcsname{\let\PY@it=\textit\def\PY@tc##1{\textcolor[rgb]{0.25,0.50,0.50}{##1}}}
\expandafter\def\csname PY@tok@cpf\endcsname{\let\PY@it=\textit\def\PY@tc##1{\textcolor[rgb]{0.25,0.50,0.50}{##1}}}
\expandafter\def\csname PY@tok@c1\endcsname{\let\PY@it=\textit\def\PY@tc##1{\textcolor[rgb]{0.25,0.50,0.50}{##1}}}
\expandafter\def\csname PY@tok@cs\endcsname{\let\PY@it=\textit\def\PY@tc##1{\textcolor[rgb]{0.25,0.50,0.50}{##1}}}

\def\PYZbs{\char`\\}
\def\PYZus{\char`\_}
\def\PYZob{\char`\{}
\def\PYZcb{\char`\}}
\def\PYZca{\char`\^}
\def\PYZam{\char`\&}
\def\PYZlt{\char`\<}
\def\PYZgt{\char`\>}
\def\PYZsh{\char`\#}
\def\PYZpc{\char`\%}
\def\PYZdl{\char`\$}
\def\PYZhy{\char`\-}
\def\PYZsq{\char`\'}
\def\PYZdq{\char`\"}
\def\PYZti{\char`\~}
% for compatibility with earlier versions
\def\PYZat{@}
\def\PYZlb{[}
\def\PYZrb{]}
\makeatother


    % Exact colors from NB
    \definecolor{incolor}{rgb}{0.0, 0.0, 0.5}
    \definecolor{outcolor}{rgb}{0.545, 0.0, 0.0}



    
    % Prevent overflowing lines due to hard-to-break entities
    \sloppy 
    % Setup hyperref package
    \hypersetup{
      breaklinks=true,  % so long urls are correctly broken across lines
      colorlinks=true,
      urlcolor=urlcolor,
      linkcolor=linkcolor,
      citecolor=citecolor,
      }
    % Slightly bigger margins than the latex defaults
    
    \geometry{verbose,tmargin=1in,bmargin=1in,lmargin=1in,rmargin=1in}
    
    

    \begin{document}
    
    
    \maketitle
    
    

    
    \section{PROGRAMMING ASSIGNMENT 1 - DMITRY
KALIKA}\label{programming-assignment-1---dmitry-kalika}

\subsection{Importing necessary
modules}\label{importing-necessary-modules}

Modules that I wrote (and didn't write) are first imported. I wrote
'inversion\_count' to count inversions, 'merge\_sort' to perform merge
sort and count the number of operations, and 'selection' to select the
lower median efficiently and count the number of operations. All other
modules were used for analysis, but not actually used to compute sort or
compute the medians.

    \begin{Verbatim}[commandchars=\\\{\}]
{\color{incolor}In [{\color{incolor}1}]:} \PY{c+c1}{\PYZsh{}Import Modules}
        \PY{k+kn}{import} \PY{n+nn}{inversion\PYZus{}count} \PY{k}{as} \PY{n+nn}{inv\PYZus{}c}  \PY{c+c1}{\PYZsh{} My inversion counter module}
        \PY{k+kn}{from} \PY{n+nn}{random} \PY{k}{import} \PY{n}{random}  \PY{c+c1}{\PYZsh{} To generate random numbers}
        \PY{k+kn}{from} \PY{n+nn}{random} \PY{k}{import} \PY{n}{randrange}  \PY{c+c1}{\PYZsh{} To generate random numbers}
        \PY{k+kn}{from} \PY{n+nn}{random} \PY{k}{import} \PY{n}{shuffle}  \PY{c+c1}{\PYZsh{} Used in inversion experiment to shuffle the array subsets}
        \PY{k+kn}{import} \PY{n+nn}{numpy} \PY{k}{as} \PY{n+nn}{np}  \PY{c+c1}{\PYZsh{} Only used to generate a linearly spaced array for various N\PYZsq{}s in simulation}
        \PY{k+kn}{import} \PY{n+nn}{merge\PYZus{}sort}  \PY{c+c1}{\PYZsh{} My merge sort module}
        \PY{k+kn}{import} \PY{n+nn}{selection}  \PY{c+c1}{\PYZsh{} My select module}
        \PY{k+kn}{import} \PY{n+nn}{statistics} \PY{k}{as} \PY{n+nn}{stats}  \PY{c+c1}{\PYZsh{} Statistics module to verify median}
        \PY{k+kn}{import} \PY{n+nn}{matplotlib}\PY{n+nn}{.}\PY{n+nn}{pyplot} \PY{k}{as} \PY{n+nn}{plt}  \PY{c+c1}{\PYZsh{} Plotting module}
        \PY{k+kn}{import} \PY{n+nn}{matplotlib}\PY{n+nn}{.}\PY{n+nn}{style} \PY{k}{as} \PY{n+nn}{sty}  \PY{c+c1}{\PYZsh{} More for plotting module}
        \PY{k+kn}{import} \PY{n+nn}{math}  \PY{c+c1}{\PYZsh{} For log function}
        \PY{n}{sty}\PY{o}{.}\PY{n}{use}\PY{p}{(}\PY{l+s+s2}{\PYZdq{}}\PY{l+s+s2}{fivethirtyeight}\PY{l+s+s2}{\PYZdq{}}\PY{p}{)}  \PY{c+c1}{\PYZsh{} Use five\PYZhy{}thirty\PYZhy{}eight blog figure style}
        \PY{o}{\PYZpc{}}\PY{k}{matplotlib} inline  
\end{Verbatim}


    \section{Merge Sort and Select
Example}\label{merge-sort-and-select-example}

First, we will show an example of merge-sort and select in action. A
random array of size N = 10000 was generated, sorted using merge-sort,
and then the lower median was computed directly from the sorted array.
The lower median was also computed using the select algorithm. The lower
medians computed directly from merge-sort, the select algorithm, and the
statistics toolbox (not written by me) were computed and compared; the
number of operations between the two algorithms written by me were
compared.

    \begin{Verbatim}[commandchars=\\\{\}]
{\color{incolor}In [{\color{incolor}2}]:} \PY{n}{n} \PY{o}{=} \PY{l+m+mi}{10000}
        \PY{n}{maxV} \PY{o}{=} \PY{l+m+mi}{100}  \PY{c+c1}{\PYZsh{}Max range of random number}
        \PY{n}{rand\PYZus{}array} \PY{o}{=} \PY{p}{[}\PY{n}{maxV}\PY{o}{*}\PY{n}{random}\PY{p}{(}\PY{p}{)} \PY{k}{for} \PY{n}{p} \PY{o+ow}{in} \PY{n+nb}{range}\PY{p}{(}\PY{l+m+mi}{0}\PY{p}{,} \PY{n}{n}\PY{p}{)}\PY{p}{]}
        \PY{n}{plt}\PY{o}{.}\PY{n}{plot}\PY{p}{(}\PY{n}{rand\PYZus{}array}\PY{p}{,}\PY{n}{marker}\PY{o}{=}\PY{l+s+s2}{\PYZdq{}}\PY{l+s+s2}{o}\PY{l+s+s2}{\PYZdq{}}\PY{p}{,}\PY{n}{linestyle}\PY{o}{=}\PY{l+s+s2}{\PYZdq{}}\PY{l+s+s2}{None}\PY{l+s+s2}{\PYZdq{}}\PY{p}{)}\PY{p}{;}
        \PY{n}{plt}\PY{o}{.}\PY{n}{plot}\PY{p}{(}\PY{n}{merge\PYZus{}sort}\PY{o}{.}\PY{n}{sort}\PY{p}{(}\PY{n}{rand\PYZus{}array}\PY{p}{)}\PY{p}{)}\PY{p}{;}
        \PY{n}{plt}\PY{o}{.}\PY{n}{legend}\PY{p}{(}\PY{p}{[}\PY{l+s+s2}{\PYZdq{}}\PY{l+s+s2}{Unsorted Values}\PY{l+s+s2}{\PYZdq{}}\PY{p}{,}\PY{l+s+s2}{\PYZdq{}}\PY{l+s+s2}{Sorted Values}\PY{l+s+s2}{\PYZdq{}}\PY{p}{]}\PY{p}{,}\PY{n}{loc}\PY{o}{=}\PY{l+s+s1}{\PYZsq{}}\PY{l+s+s1}{lower right}\PY{l+s+s1}{\PYZsq{}}\PY{p}{)}\PY{p}{;}
        \PY{n}{plt}\PY{o}{.}\PY{n}{xlabel}\PY{p}{(}\PY{l+s+s2}{\PYZdq{}}\PY{l+s+s2}{Value Index}\PY{l+s+s2}{\PYZdq{}}\PY{p}{)}\PY{p}{;}
        \PY{n}{plt}\PY{o}{.}\PY{n}{ylabel}\PY{p}{(}\PY{l+s+s2}{\PYZdq{}}\PY{l+s+s2}{Value}\PY{l+s+s2}{\PYZdq{}}\PY{p}{)}\PY{p}{;}
\end{Verbatim}


    \begin{center}
    \adjustimage{max size={0.9\linewidth}{0.9\paperheight}}{output_3_0.png}
    \end{center}
    { \hspace*{\fill} \\}
    
    \begin{Verbatim}[commandchars=\\\{\}]
{\color{incolor}In [{\color{incolor}3}]:} \PY{n}{sorted\PYZus{}array}\PY{p}{,}\PY{n}{merge\PYZus{}sort\PYZus{}ops} \PY{o}{=} \PY{n}{merge\PYZus{}sort}\PY{o}{.}\PY{n}{sort}\PY{p}{(}\PY{n}{rand\PYZus{}array}\PY{p}{,}\PY{n}{out\PYZus{}ops} \PY{o}{=} \PY{k+kc}{True}\PY{p}{)}  \PY{c+c1}{\PYZsh{}Do merge sort}
        
        \PY{n+nb}{print}\PY{p}{(}\PY{l+s+s2}{\PYZdq{}}\PY{l+s+s2}{Median from Merge Sort: }\PY{l+s+s2}{\PYZdq{}} \PY{o}{+} \PY{n+nb}{repr}\PY{p}{(}\PY{n}{sorted\PYZus{}array}\PY{p}{[}\PY{n}{math}\PY{o}{.}\PY{n}{ceil}\PY{p}{(}\PY{n+nb}{len}\PY{p}{(}\PY{n}{sorted\PYZus{}array}\PY{p}{)}\PY{o}{/}\PY{l+m+mi}{2}\PY{p}{)}\PY{o}{\PYZhy{}}\PY{l+m+mi}{1}\PY{p}{]}\PY{p}{)}\PY{p}{)}
        \PY{n+nb}{print}\PY{p}{(}\PY{l+s+s2}{\PYZdq{}}\PY{l+s+s2}{Number of operations for Merge Sort: }\PY{l+s+s2}{\PYZdq{}}\PY{o}{+}\PY{n+nb}{repr}\PY{p}{(}\PY{n}{merge\PYZus{}sort\PYZus{}ops}\PY{p}{)}\PY{p}{)}
        \PY{n+nb}{print}\PY{p}{(}\PY{l+s+s2}{\PYZdq{}}\PY{l+s+s2}{\PYZdq{}}\PY{p}{)}
        \PY{n}{select\PYZus{}median}\PY{p}{,}\PY{n}{select\PYZus{}ops} \PY{o}{=} \PY{n}{selection}\PY{o}{.}\PY{n}{select}\PY{p}{(}\PY{n}{rand\PYZus{}array}\PY{p}{,}\PY{n}{out\PYZus{}ops} \PY{o}{=} \PY{k+kc}{True}\PY{p}{)}
        \PY{n+nb}{print}\PY{p}{(}\PY{l+s+s2}{\PYZdq{}}\PY{l+s+s2}{Median from Select: }\PY{l+s+s2}{\PYZdq{}} \PY{o}{+} \PY{n+nb}{repr}\PY{p}{(}\PY{n}{select\PYZus{}median}\PY{p}{)}\PY{p}{)}
        \PY{n+nb}{print}\PY{p}{(}\PY{l+s+s2}{\PYZdq{}}\PY{l+s+s2}{Number of operations for Select: }\PY{l+s+s2}{\PYZdq{}}\PY{o}{+}\PY{n+nb}{repr}\PY{p}{(}\PY{n}{select\PYZus{}ops}\PY{p}{)}\PY{p}{)}
        \PY{n+nb}{print}\PY{p}{(}\PY{l+s+s2}{\PYZdq{}}\PY{l+s+s2}{\PYZdq{}}\PY{p}{)}
        \PY{n+nb}{print}\PY{p}{(}\PY{l+s+s2}{\PYZdq{}}\PY{l+s+s2}{Median from python built\PYZhy{}in statistics module: }\PY{l+s+s2}{\PYZdq{}}\PY{p}{,}\PY{n+nb}{repr}\PY{p}{(}\PY{n}{stats}\PY{o}{.}\PY{n}{median\PYZus{}low}\PY{p}{(}\PY{n}{rand\PYZus{}array}\PY{p}{)}\PY{p}{)}\PY{p}{)}
\end{Verbatim}


    \begin{Verbatim}[commandchars=\\\{\}]
Median from Merge Sort: 49.919888824571245
Number of operations for Merge Sort: 120446

Median from Select: 49.919888824571245
Number of operations for Select: 36511

Median from python built-in statistics module:  49.919888824571245

    \end{Verbatim}

    \subsection{Discussion}\label{discussion}

All algorithms compute the same lower median! The number of operations
required by the select algorithm is significantly fewer than performing
the entire merge-sort

These results are on par with what was expected, however, we still need
to validate that merge-sort is sorting accurately, and both merge-sort
and select algorithms are computing the lower median.

\subsection{Algorithm Verification}\label{algorithm-verification}

To verify that the algorithms are working as expected, the merge-sort
algorithm and select algorithm are tested on arrays of various sizes.
Merge-sort is first validated by making sure that its results match that
of the python built-in sorting algorithm. Then, if merge-sort is
validated, computing the lower median directly is guaranteed to give us
the correct lower median - this lower median is then used to validate
the lower median computed by select.

    \begin{Verbatim}[commandchars=\\\{\}]
{\color{incolor}In [{\color{incolor}4}]:} \PY{n}{max\PYZus{}array\PYZus{}size} \PY{o}{=} \PY{l+m+mi}{1000}  \PY{c+c1}{\PYZsh{} Max array size}
        \PY{n}{max\PYZus{}value\PYZus{}size} \PY{o}{=} \PY{l+m+mi}{100}  \PY{c+c1}{\PYZsh{} Max value of an element in the array}
        \PY{n}{n\PYZus{}test} \PY{o}{=} \PY{l+m+mi}{10000}  \PY{c+c1}{\PYZsh{} Runs}
        \PY{n}{array\PYZus{}size} \PY{o}{=} \PY{p}{[}\PY{n}{math}\PY{o}{.}\PY{n}{ceil}\PY{p}{(}\PY{n}{max\PYZus{}array\PYZus{}size}\PY{o}{*}\PY{n}{random}\PY{p}{(}\PY{p}{)}\PY{p}{)} \PY{k}{for} \PY{n}{p} \PY{o+ow}{in} \PY{n+nb}{range}\PY{p}{(}\PY{l+m+mi}{0}\PY{p}{,} \PY{n}{n\PYZus{}test}\PY{p}{)}\PY{p}{]}  \PY{c+c1}{\PYZsh{} randomly select size of arrays}
        
        \PY{n}{sorted\PYZus{}fail} \PY{o}{=} \PY{l+m+mi}{0}  \PY{c+c1}{\PYZsh{} Initialize with 0 sorted failures}
        \PY{n}{median\PYZus{}fail} \PY{o}{=} \PY{l+m+mi}{0}  \PY{c+c1}{\PYZsh{} Initialize with 0 median failures}
        
        \PY{k}{for} \PY{n}{c\PYZus{}array\PYZus{}size} \PY{o+ow}{in} \PY{n}{array\PYZus{}size}\PY{p}{:}
            \PY{n}{rand\PYZus{}array} \PY{o}{=} \PY{p}{[}\PY{n}{max\PYZus{}value\PYZus{}size}\PY{o}{*}\PY{n}{random}\PY{p}{(}\PY{p}{)} \PY{k}{for} \PY{n}{p} \PY{o+ow}{in} \PY{n+nb}{range}\PY{p}{(}\PY{l+m+mi}{0}\PY{p}{,} \PY{n}{c\PYZus{}array\PYZus{}size}\PY{p}{)}\PY{p}{]}  \PY{c+c1}{\PYZsh{} Generate array of random size}
            
            \PY{n}{merge\PYZus{}sorted} \PY{o}{=} \PY{n}{merge\PYZus{}sort}\PY{o}{.}\PY{n}{sort}\PY{p}{(}\PY{n}{rand\PYZus{}array}\PY{p}{,}\PY{n}{out\PYZus{}ops} \PY{o}{=} \PY{k+kc}{False}\PY{p}{)}  \PY{c+c1}{\PYZsh{} Do merge sort}
            \PY{n}{select\PYZus{}median} \PY{o}{=} \PY{n}{selection}\PY{o}{.}\PY{n}{select}\PY{p}{(}\PY{n}{rand\PYZus{}array}\PY{p}{,}\PY{n}{out\PYZus{}ops} \PY{o}{=} \PY{k+kc}{False}\PY{p}{)}  \PY{c+c1}{\PYZsh{} Do select}
           
            \PY{n}{rand\PYZus{}array}\PY{o}{.}\PY{n}{sort}\PY{p}{(}\PY{p}{)}  \PY{c+c1}{\PYZsh{} Do built in sort}
            \PY{k}{if} \PY{o+ow}{not} \PY{p}{(}\PY{n}{merge\PYZus{}sorted} \PY{o}{==} \PY{n}{rand\PYZus{}array}\PY{p}{)}\PY{p}{:}  \PY{c+c1}{\PYZsh{} Count the number of times sort failed}
                \PY{n}{sorted\PYZus{}fail} \PY{o}{+}\PY{o}{=} \PY{l+m+mi}{1}  \PY{c+c1}{\PYZsh{} Add 1 if sorting failed}
            
            \PY{k}{if} \PY{o+ow}{not} \PY{p}{(}\PY{n}{merge\PYZus{}sorted}\PY{p}{[}\PY{n}{math}\PY{o}{.}\PY{n}{ceil}\PY{p}{(}\PY{n+nb}{len}\PY{p}{(}\PY{n}{merge\PYZus{}sorted}\PY{p}{)}\PY{o}{/}\PY{l+m+mi}{2}\PY{p}{)}\PY{o}{\PYZhy{}}\PY{l+m+mi}{1}\PY{p}{]}\PY{o}{==}\PY{n}{select\PYZus{}median}\PY{p}{)}\PY{p}{:}
                \PY{n}{median\PYZus{}fail} \PY{o}{+}\PY{o}{=}\PY{l+m+mi}{1}  \PY{c+c1}{\PYZsh{} Add 1 if lower median calculation failed}
\end{Verbatim}


    \begin{Verbatim}[commandchars=\\\{\}]
{\color{incolor}In [{\color{incolor}5}]:} \PY{n+nb}{print}\PY{p}{(}\PY{l+s+s2}{\PYZdq{}}\PY{l+s+s2}{Sorting Failed: }\PY{l+s+s2}{\PYZdq{}} \PY{o}{+} \PY{n+nb}{repr}\PY{p}{(}\PY{n}{sorted\PYZus{}fail}\PY{p}{)} \PY{o}{+} \PY{l+s+s2}{\PYZdq{}}\PY{l+s+s2}{ times}\PY{l+s+s2}{\PYZdq{}}\PY{p}{)}
        \PY{n+nb}{print}\PY{p}{(}\PY{l+s+s2}{\PYZdq{}}\PY{l+s+s2}{Median Failed: }\PY{l+s+s2}{\PYZdq{}} \PY{o}{+} \PY{n+nb}{repr}\PY{p}{(}\PY{n}{median\PYZus{}fail}\PY{p}{)} \PY{o}{+} \PY{l+s+s2}{\PYZdq{}}\PY{l+s+s2}{ times}\PY{l+s+s2}{\PYZdq{}}\PY{p}{)}
\end{Verbatim}


    \begin{Verbatim}[commandchars=\\\{\}]
Sorting Failed: 0 times
Median Failed: 0 times

    \end{Verbatim}

    \subsection{Discussion}\label{discussion}

The results above shows that all 10,000 randomly generated arrays were
correctly sorted, and all merge-sort lower medians match that of the
select algorithm. This gives us confidence that our merge-sort and
select algorithms are doing exactly what we expect them to do.

    \section{Algorithm Complexity
Experiments}\label{algorithm-complexity-experiments}

In order to get a good idea about the number of operations required to
find the lower median as a function of N, we'll compute 10 medians (each
from a random array) for N = 1 to 100000. Computing these 10 medians for
all 100000 values of N would have taken a really long time, so instead
we will look at 100 values of N, linearly spaced between 1 and 100000.
For each N, we'll find the average number of operations required to
compute the median using both the merge-sort and selection algorithms.
We'll then plot a comparison.

    \begin{Verbatim}[commandchars=\\\{\}]
{\color{incolor}In [{\color{incolor}6}]:} \PY{n}{nRuns} \PY{o}{=} \PY{l+m+mi}{10}
        \PY{n}{N} \PY{o}{=} \PY{n}{np}\PY{o}{.}\PY{n}{linspace}\PY{p}{(}\PY{l+m+mi}{1}\PY{p}{,}\PY{l+m+mi}{100000}\PY{p}{,}\PY{n}{num}\PY{o}{=}\PY{l+m+mi}{100}\PY{p}{,}\PY{n}{dtype}\PY{o}{=}\PY{l+s+s1}{\PYZsq{}}\PY{l+s+s1}{uint64}\PY{l+s+s1}{\PYZsq{}}\PY{p}{)}  \PY{c+c1}{\PYZsh{} Linearly spaced array of size 100 from n = 1 to 100,000}
        
        \PY{n}{merge\PYZus{}sort\PYZus{}avg\PYZus{}ops} \PY{o}{=} \PY{p}{[}\PY{p}{]}  \PY{c+c1}{\PYZsh{} List to hold average ops as a function of N}
        \PY{n}{select\PYZus{}avg\PYZus{}ops} \PY{o}{=} \PY{p}{[}\PY{p}{]}  \PY{c+c1}{\PYZsh{} List to hold average ops as a function of N}
        
        \PY{k}{for} \PY{n}{iN} \PY{o+ow}{in} \PY{n}{N}\PY{p}{:}
            \PY{n}{c\PYZus{}merge\PYZus{}ops} \PY{o}{=} \PY{l+m+mi}{0}
            \PY{n}{c\PYZus{}select\PYZus{}ops} \PY{o}{=} \PY{l+m+mi}{0}
            \PY{k}{for} \PY{n}{iRuns} \PY{o+ow}{in} \PY{n+nb}{range}\PY{p}{(}\PY{l+m+mi}{1}\PY{p}{,}\PY{n}{nRuns}\PY{o}{+}\PY{l+m+mi}{1}\PY{p}{)}\PY{p}{:}
                \PY{n}{rand\PYZus{}array} \PY{o}{=} \PY{p}{[}\PY{n}{maxV}\PY{o}{*}\PY{n}{random}\PY{p}{(}\PY{p}{)} \PY{k}{for} \PY{n}{p} \PY{o+ow}{in} \PY{n+nb}{range}\PY{p}{(}\PY{l+m+mi}{0}\PY{p}{,} \PY{n}{iN}\PY{p}{)}\PY{p}{]}  \PY{c+c1}{\PYZsh{} Generate random array of size N}
                \PY{n}{\PYZus{}}\PY{p}{,}\PY{n}{merge\PYZus{}sort\PYZus{}ops} \PY{o}{=} \PY{n}{merge\PYZus{}sort}\PY{o}{.}\PY{n}{sort}\PY{p}{(}\PY{n}{rand\PYZus{}array}\PY{p}{,}\PY{n}{out\PYZus{}ops} \PY{o}{=} \PY{k+kc}{True}\PY{p}{)}  \PY{c+c1}{\PYZsh{} Compute merge\PYZhy{}sort operations}
                \PY{n}{\PYZus{}}\PY{p}{,}\PY{n}{select\PYZus{}ops} \PY{o}{=} \PY{n}{selection}\PY{o}{.}\PY{n}{select}\PY{p}{(}\PY{n}{rand\PYZus{}array}\PY{p}{,}\PY{n}{out\PYZus{}ops} \PY{o}{=} \PY{k+kc}{True}\PY{p}{)}  \PY{c+c1}{\PYZsh{} Compute select operations}
                \PY{n}{c\PYZus{}merge\PYZus{}ops}\PY{o}{+}\PY{o}{=}\PY{n}{merge\PYZus{}sort\PYZus{}ops}
                \PY{n}{c\PYZus{}select\PYZus{}ops}\PY{o}{+}\PY{o}{=}\PY{n}{select\PYZus{}ops}
            \PY{n}{merge\PYZus{}sort\PYZus{}avg\PYZus{}ops}\PY{o}{.}\PY{n}{append}\PY{p}{(}\PY{n}{c\PYZus{}merge\PYZus{}ops}\PY{o}{/}\PY{n}{nRuns}\PY{p}{)}  \PY{c+c1}{\PYZsh{} Find average \PYZsh{} of ops for merge\PYZhy{}sort for array of size N}
            \PY{n}{select\PYZus{}avg\PYZus{}ops}\PY{o}{.}\PY{n}{append}\PY{p}{(}\PY{n}{c\PYZus{}select\PYZus{}ops}\PY{o}{/}\PY{n}{nRuns}\PY{p}{)}  \PY{c+c1}{\PYZsh{} Find average \PYZsh{} of ops for select for array of size N}
\end{Verbatim}


    \begin{Verbatim}[commandchars=\\\{\}]
{\color{incolor}In [{\color{incolor}7}]:} \PY{n}{plt}\PY{o}{.}\PY{n}{plot}\PY{p}{(}\PY{n}{N}\PY{p}{,}\PY{n}{merge\PYZus{}sort\PYZus{}avg\PYZus{}ops}\PY{p}{)}\PY{p}{;}  \PY{c+c1}{\PYZsh{} Plot average ops as a function of N using merge\PYZhy{}sort}
        \PY{n}{plt}\PY{o}{.}\PY{n}{plot}\PY{p}{(}\PY{n}{N}\PY{p}{,}\PY{n}{select\PYZus{}avg\PYZus{}ops}\PY{p}{)}\PY{p}{;}  \PY{c+c1}{\PYZsh{} Plot average ops as a function of N using select}
        \PY{n}{plt}\PY{o}{.}\PY{n}{legend}\PY{p}{(}\PY{p}{[}\PY{l+s+s2}{\PYZdq{}}\PY{l+s+s2}{Merge\PYZhy{}sort}\PY{l+s+s2}{\PYZdq{}}\PY{p}{,}\PY{l+s+s2}{\PYZdq{}}\PY{l+s+s2}{Select}\PY{l+s+s2}{\PYZdq{}}\PY{p}{]}\PY{p}{,}\PY{n}{loc}\PY{o}{=}\PY{l+s+s2}{\PYZdq{}}\PY{l+s+s2}{center right}\PY{l+s+s2}{\PYZdq{}}\PY{p}{)}\PY{p}{;}
        \PY{n}{plt}\PY{o}{.}\PY{n}{xlabel}\PY{p}{(}\PY{l+s+s2}{\PYZdq{}}\PY{l+s+s2}{N (Size of Array)}\PY{l+s+s2}{\PYZdq{}}\PY{p}{)}\PY{p}{;}
        \PY{n}{plt}\PY{o}{.}\PY{n}{ylabel}\PY{p}{(}\PY{l+s+s2}{\PYZdq{}}\PY{l+s+s2}{Operations}\PY{l+s+s2}{\PYZdq{}}\PY{p}{)}\PY{p}{;}
        \PY{n}{plt}\PY{o}{.}\PY{n}{title}\PY{p}{(}\PY{l+s+s1}{\PYZsq{}}\PY{l+s+s1}{Operations Comparison Between Select and Merge\PYZhy{}sort}\PY{l+s+s1}{\PYZsq{}}\PY{p}{)}\PY{p}{;}
\end{Verbatim}


    \begin{center}
    \adjustimage{max size={0.9\linewidth}{0.9\paperheight}}{output_11_0.png}
    \end{center}
    { \hspace*{\fill} \\}
    
    The results above show a larger-scale simulation to compare the number
of operations required for merge-sort and select algorithms. Though we
haven't explicitely performed a linear or exponential curve fit, it
appears from the plot that the select curve is linear, and the
merge-sort curve is somewhere between linear and exponential ( perhaps
N*log(N))?

We can figure out what kind of average complexities the algorithms are
by computing C as a function of N for a given complexity (if we go by
the notation that \(f(n) = O(g(N))\) is equivalent to \(f(n)\leq Cg(n)\)
at some \(n>k\)). If C remains relatively constant as a function of N,
then we know that \(f(n) = Cg(n)\). We will do this for complexities
\(N\), \(NlogN\), and \(N^2\):

If linear complexity: \(C = \dfrac{ops}{N}\)

If \(N\log(N)\) complexity: \(C = \dfrac{ops}{N\log(N)}\)

If \(N^2\) complexity: \(C = \dfrac{ops}{N^2}\)

Note that \(log(C)\) was plotted instead of \(C\) - this was done
because the C value can be very low and difficult to visualize,
especially for the \(N^2\) complexity test. However, If C is constant,
then \(log(C)\) will also be constant!

Another thing to note is that this method is similar to doing regression
- if the value C does not change, then \(f(n) = Cg(n)\) at that fixed C.

    \begin{Verbatim}[commandchars=\\\{\}]
{\color{incolor}In [{\color{incolor}8}]:} \PY{c+c1}{\PYZsh{} C if N}
        \PY{n}{c\PYZus{}merge} \PY{o}{=} \PY{p}{[}\PY{p}{]}
        \PY{n}{c\PYZus{}select} \PY{o}{=} \PY{p}{[}\PY{p}{]}
        \PY{n+nb}{len}\PY{p}{(}\PY{n}{merge\PYZus{}sort\PYZus{}avg\PYZus{}ops}\PY{p}{)}
        \PY{k}{for} \PY{n}{i} \PY{o+ow}{in} \PY{n+nb}{range}\PY{p}{(}\PY{n+nb}{len}\PY{p}{(}\PY{n}{N}\PY{p}{)}\PY{o}{\PYZhy{}}\PY{l+m+mi}{1}\PY{p}{)}\PY{p}{:}
            \PY{n}{c\PYZus{}merge}\PY{o}{.}\PY{n}{append}\PY{p}{(}\PY{n}{math}\PY{o}{.}\PY{n}{log}\PY{p}{(}\PY{n}{merge\PYZus{}sort\PYZus{}avg\PYZus{}ops}\PY{p}{[}\PY{n}{i}\PY{o}{+}\PY{l+m+mi}{1}\PY{p}{]}\PY{o}{/}\PY{p}{(}\PY{n}{N}\PY{p}{[}\PY{n}{i}\PY{o}{+}\PY{l+m+mi}{1}\PY{p}{]}\PY{p}{)}\PY{p}{)}\PY{p}{)}
            \PY{n}{c\PYZus{}select}\PY{o}{.}\PY{n}{append}\PY{p}{(}\PY{n}{math}\PY{o}{.}\PY{n}{log}\PY{p}{(}\PY{n}{select\PYZus{}avg\PYZus{}ops}\PY{p}{[}\PY{n}{i}\PY{o}{+}\PY{l+m+mi}{1}\PY{p}{]}\PY{o}{/}\PY{p}{(}\PY{n}{N}\PY{p}{[}\PY{n}{i}\PY{o}{+}\PY{l+m+mi}{1}\PY{p}{]}\PY{p}{)}\PY{p}{)}\PY{p}{)}
            
        \PY{n}{plt}\PY{o}{.}\PY{n}{plot}\PY{p}{(}\PY{n}{N}\PY{p}{,}\PY{p}{[}\PY{k+kc}{None}\PY{p}{]}\PY{o}{+}\PY{n}{c\PYZus{}merge}\PY{p}{)}\PY{p}{;}  \PY{c+c1}{\PYZsh{} Plot average ops as a function of N using merge\PYZhy{}sort}
        \PY{n}{plt}\PY{o}{.}\PY{n}{plot}\PY{p}{(}\PY{n}{N}\PY{p}{,}\PY{p}{[}\PY{k+kc}{None}\PY{p}{]}\PY{o}{+}\PY{n}{c\PYZus{}select}\PY{p}{)}\PY{p}{;}  \PY{c+c1}{\PYZsh{} Plot average ops as a function of N using select}
        \PY{n}{plt}\PY{o}{.}\PY{n}{legend}\PY{p}{(}\PY{p}{[}\PY{l+s+s2}{\PYZdq{}}\PY{l+s+s2}{Merge\PYZhy{}sort}\PY{l+s+s2}{\PYZdq{}}\PY{p}{,}\PY{l+s+s2}{\PYZdq{}}\PY{l+s+s2}{Select}\PY{l+s+s2}{\PYZdq{}}\PY{p}{]}\PY{p}{,}\PY{n}{loc}\PY{o}{=}\PY{l+s+s2}{\PYZdq{}}\PY{l+s+s2}{center right}\PY{l+s+s2}{\PYZdq{}}\PY{p}{)}\PY{p}{;}
        \PY{n}{plt}\PY{o}{.}\PY{n}{xlabel}\PY{p}{(}\PY{l+s+s1}{\PYZsq{}}\PY{l+s+s1}{N}\PY{l+s+s1}{\PYZsq{}}\PY{p}{)}\PY{p}{;}
        \PY{n}{plt}\PY{o}{.}\PY{n}{ylabel}\PY{p}{(}\PY{l+s+s1}{\PYZsq{}}\PY{l+s+s1}{log(C)}\PY{l+s+s1}{\PYZsq{}}\PY{p}{)}\PY{p}{;}
        \PY{n}{plt}\PY{o}{.}\PY{n}{title}\PY{p}{(}\PY{l+s+s1}{\PYZsq{}}\PY{l+s+s1}{N Complexity Test}\PY{l+s+s1}{\PYZsq{}}\PY{p}{)}\PY{p}{;}
\end{Verbatim}


    \begin{center}
    \adjustimage{max size={0.9\linewidth}{0.9\paperheight}}{output_13_0.png}
    \end{center}
    { \hspace*{\fill} \\}
    
    C is constant for the select algorithm.

C is NOT constant for the merge-sort algorithm.

    \begin{Verbatim}[commandchars=\\\{\}]
{\color{incolor}In [{\color{incolor}9}]:} \PY{c+c1}{\PYZsh{} C if NlogN}
        \PY{n}{c\PYZus{}merge} \PY{o}{=} \PY{p}{[}\PY{p}{]}
        \PY{n}{c\PYZus{}select} \PY{o}{=} \PY{p}{[}\PY{p}{]}
        \PY{n+nb}{len}\PY{p}{(}\PY{n}{merge\PYZus{}sort\PYZus{}avg\PYZus{}ops}\PY{p}{)}
        \PY{k}{for} \PY{n}{i} \PY{o+ow}{in} \PY{n+nb}{range}\PY{p}{(}\PY{n+nb}{len}\PY{p}{(}\PY{n}{N}\PY{p}{)}\PY{o}{\PYZhy{}}\PY{l+m+mi}{1}\PY{p}{)}\PY{p}{:}
            \PY{n}{c\PYZus{}merge}\PY{o}{.}\PY{n}{append}\PY{p}{(}\PY{n}{math}\PY{o}{.}\PY{n}{log}\PY{p}{(}\PY{n}{merge\PYZus{}sort\PYZus{}avg\PYZus{}ops}\PY{p}{[}\PY{n}{i}\PY{o}{+}\PY{l+m+mi}{1}\PY{p}{]}\PY{o}{/}\PY{p}{(}\PY{n}{N}\PY{p}{[}\PY{n}{i}\PY{o}{+}\PY{l+m+mi}{1}\PY{p}{]}\PY{o}{*}\PY{n}{math}\PY{o}{.}\PY{n}{log2}\PY{p}{(}\PY{n}{N}\PY{p}{[}\PY{n}{i}\PY{o}{+}\PY{l+m+mi}{1}\PY{p}{]}\PY{p}{)}\PY{p}{)}\PY{p}{)}\PY{p}{)}
            \PY{n}{c\PYZus{}select}\PY{o}{.}\PY{n}{append}\PY{p}{(}\PY{n}{math}\PY{o}{.}\PY{n}{log}\PY{p}{(}\PY{n}{select\PYZus{}avg\PYZus{}ops}\PY{p}{[}\PY{n}{i}\PY{o}{+}\PY{l+m+mi}{1}\PY{p}{]}\PY{o}{/}\PY{p}{(}\PY{n}{N}\PY{p}{[}\PY{n}{i}\PY{o}{+}\PY{l+m+mi}{1}\PY{p}{]}\PY{o}{*}\PY{n}{math}\PY{o}{.}\PY{n}{log2}\PY{p}{(}\PY{n}{N}\PY{p}{[}\PY{n}{i}\PY{o}{+}\PY{l+m+mi}{1}\PY{p}{]}\PY{p}{)}\PY{p}{)}\PY{p}{)}\PY{p}{)}
            
        \PY{n}{plt}\PY{o}{.}\PY{n}{plot}\PY{p}{(}\PY{n}{N}\PY{p}{,}\PY{p}{[}\PY{k+kc}{None}\PY{p}{]}\PY{o}{+}\PY{n}{c\PYZus{}merge}\PY{p}{)}\PY{p}{;}  \PY{c+c1}{\PYZsh{} Plot average ops as a function of N using merge\PYZhy{}sort}
        \PY{n}{plt}\PY{o}{.}\PY{n}{plot}\PY{p}{(}\PY{n}{N}\PY{p}{,}\PY{p}{[}\PY{k+kc}{None}\PY{p}{]}\PY{o}{+}\PY{n}{c\PYZus{}select}\PY{p}{)}\PY{p}{;}  \PY{c+c1}{\PYZsh{} Plot average ops as a function of N using select}
        \PY{n}{plt}\PY{o}{.}\PY{n}{legend}\PY{p}{(}\PY{p}{[}\PY{l+s+s2}{\PYZdq{}}\PY{l+s+s2}{Merge\PYZhy{}sort}\PY{l+s+s2}{\PYZdq{}}\PY{p}{,}\PY{l+s+s2}{\PYZdq{}}\PY{l+s+s2}{Select}\PY{l+s+s2}{\PYZdq{}}\PY{p}{]}\PY{p}{,}\PY{n}{loc}\PY{o}{=}\PY{l+s+s2}{\PYZdq{}}\PY{l+s+s2}{center right}\PY{l+s+s2}{\PYZdq{}}\PY{p}{)}\PY{p}{;}
        \PY{n}{plt}\PY{o}{.}\PY{n}{xlabel}\PY{p}{(}\PY{l+s+s1}{\PYZsq{}}\PY{l+s+s1}{N}\PY{l+s+s1}{\PYZsq{}}\PY{p}{)}\PY{p}{;}
        \PY{n}{plt}\PY{o}{.}\PY{n}{ylabel}\PY{p}{(}\PY{l+s+s1}{\PYZsq{}}\PY{l+s+s1}{log(C)}\PY{l+s+s1}{\PYZsq{}}\PY{p}{)}\PY{p}{;}
        \PY{n}{plt}\PY{o}{.}\PY{n}{title}\PY{p}{(}\PY{l+s+s1}{\PYZsq{}}\PY{l+s+s1}{Nlog(N) Complexity Test}\PY{l+s+s1}{\PYZsq{}}\PY{p}{)}\PY{p}{;}
\end{Verbatim}


    \begin{center}
    \adjustimage{max size={0.9\linewidth}{0.9\paperheight}}{output_15_0.png}
    \end{center}
    { \hspace*{\fill} \\}
    
    C is NOT constant for the select algorithm.

C is constant for the merge-sort algorithm.

    \begin{Verbatim}[commandchars=\\\{\}]
{\color{incolor}In [{\color{incolor}10}]:} \PY{c+c1}{\PYZsh{} C if N\PYZca{}2}
         \PY{n}{c\PYZus{}merge} \PY{o}{=} \PY{p}{[}\PY{p}{]}
         \PY{n}{c\PYZus{}select} \PY{o}{=} \PY{p}{[}\PY{p}{]}
         \PY{n+nb}{len}\PY{p}{(}\PY{n}{merge\PYZus{}sort\PYZus{}avg\PYZus{}ops}\PY{p}{)}
         \PY{k}{for} \PY{n}{i} \PY{o+ow}{in} \PY{n+nb}{range}\PY{p}{(}\PY{n+nb}{len}\PY{p}{(}\PY{n}{N}\PY{p}{)}\PY{o}{\PYZhy{}}\PY{l+m+mi}{1}\PY{p}{)}\PY{p}{:}
             \PY{n}{c\PYZus{}merge}\PY{o}{.}\PY{n}{append}\PY{p}{(}\PY{n}{math}\PY{o}{.}\PY{n}{log}\PY{p}{(}\PY{n}{merge\PYZus{}sort\PYZus{}avg\PYZus{}ops}\PY{p}{[}\PY{n}{i}\PY{o}{+}\PY{l+m+mi}{1}\PY{p}{]}\PY{o}{/}\PY{p}{(}\PY{n}{N}\PY{p}{[}\PY{n}{i}\PY{o}{+}\PY{l+m+mi}{1}\PY{p}{]}\PY{o}{*}\PY{n}{N}\PY{p}{[}\PY{n}{i}\PY{o}{+}\PY{l+m+mi}{1}\PY{p}{]}\PY{p}{)}\PY{p}{)}\PY{p}{)}
             \PY{n}{c\PYZus{}select}\PY{o}{.}\PY{n}{append}\PY{p}{(}\PY{n}{math}\PY{o}{.}\PY{n}{log}\PY{p}{(}\PY{n}{select\PYZus{}avg\PYZus{}ops}\PY{p}{[}\PY{n}{i}\PY{o}{+}\PY{l+m+mi}{1}\PY{p}{]}\PY{o}{/}\PY{p}{(}\PY{n}{N}\PY{p}{[}\PY{n}{i}\PY{o}{+}\PY{l+m+mi}{1}\PY{p}{]}\PY{o}{*}\PY{n}{N}\PY{p}{[}\PY{n}{i}\PY{o}{+}\PY{l+m+mi}{1}\PY{p}{]}\PY{p}{)}\PY{p}{)}\PY{p}{)}
             
         \PY{n}{plt}\PY{o}{.}\PY{n}{plot}\PY{p}{(}\PY{n}{N}\PY{p}{,}\PY{p}{[}\PY{k+kc}{None}\PY{p}{]}\PY{o}{+}\PY{n}{c\PYZus{}merge}\PY{p}{)}\PY{p}{;}  \PY{c+c1}{\PYZsh{} Plot average ops as a function of N using merge\PYZhy{}sort}
         \PY{n}{plt}\PY{o}{.}\PY{n}{plot}\PY{p}{(}\PY{n}{N}\PY{p}{,}\PY{p}{[}\PY{k+kc}{None}\PY{p}{]}\PY{o}{+}\PY{n}{c\PYZus{}select}\PY{p}{)}\PY{p}{;}  \PY{c+c1}{\PYZsh{} Plot average ops as a function of N using select}
         \PY{n}{plt}\PY{o}{.}\PY{n}{legend}\PY{p}{(}\PY{p}{[}\PY{l+s+s2}{\PYZdq{}}\PY{l+s+s2}{Merge\PYZhy{}sort}\PY{l+s+s2}{\PYZdq{}}\PY{p}{,}\PY{l+s+s2}{\PYZdq{}}\PY{l+s+s2}{Select}\PY{l+s+s2}{\PYZdq{}}\PY{p}{]}\PY{p}{,}\PY{n}{loc}\PY{o}{=}\PY{l+s+s2}{\PYZdq{}}\PY{l+s+s2}{center right}\PY{l+s+s2}{\PYZdq{}}\PY{p}{)}\PY{p}{;}
         \PY{n}{plt}\PY{o}{.}\PY{n}{xlabel}\PY{p}{(}\PY{l+s+s1}{\PYZsq{}}\PY{l+s+s1}{N}\PY{l+s+s1}{\PYZsq{}}\PY{p}{)}\PY{p}{;}
         \PY{n}{plt}\PY{o}{.}\PY{n}{ylabel}\PY{p}{(}\PY{l+s+s1}{\PYZsq{}}\PY{l+s+s1}{log(C)}\PY{l+s+s1}{\PYZsq{}}\PY{p}{)}\PY{p}{;}
         \PY{n}{plt}\PY{o}{.}\PY{n}{title}\PY{p}{(}\PY{l+s+s1}{\PYZsq{}}\PY{l+s+s1}{\PYZdl{}N\PYZca{}2\PYZdl{} Complexity Test}\PY{l+s+s1}{\PYZsq{}}\PY{p}{)}\PY{p}{;}
\end{Verbatim}


    \begin{center}
    \adjustimage{max size={0.9\linewidth}{0.9\paperheight}}{output_17_0.png}
    \end{center}
    { \hspace*{\fill} \\}
    
    C is NOT constant for the select algorithm.

C is NOT constant for the merge-sort algorithm.

    \subsection{Discussion}\label{discussion}

These 3 plots show that the select algorithm likely has N complexity,
and the merge-sort has NlogN complexity. We were also able to show that
neither of the algorithms have a complexity of \(N^2\). It is impressive
that the C values are constant, even when we took the average of only 10
median selects at each N - I expected more jitter in the curves.

    \subsection{Inversion Ratio
Experiment}\label{inversion-ratio-experiment}

In the previou section, the inversion ratio of the input array was not
considered when computing the number of operations required to compute
the lower median. In this experiment, we won't vary the size of the
random array (n=1000), however, we will instead vary the number of
inversions. To do this, we will generate 10,000 random arrays with
various inversion scores (0 = no inversions, 1 = max inversions).
Hopefully, we will get the entire range of inversions since we generated
so many random arrays! We'll see how the number of inversions affects
the required number of operations for both merge-sort and select.

    \begin{Verbatim}[commandchars=\\\{\}]
{\color{incolor}In [{\color{incolor}11}]:} \PY{n}{n} \PY{o}{=} \PY{l+m+mi}{1000}
         \PY{n}{n\PYZus{}runs} \PY{o}{=} \PY{l+m+mi}{200}
         
         \PY{n}{all\PYZus{}inversions} \PY{o}{=} \PY{p}{[}\PY{p}{]}
         \PY{n}{all\PYZus{}merge\PYZus{}sort\PYZus{}ops} \PY{o}{=} \PY{p}{[}\PY{p}{]}
         \PY{n}{all\PYZus{}select\PYZus{}ops} \PY{o}{=} \PY{p}{[}\PY{p}{]}
         
         \PY{k}{for} \PY{n}{run} \PY{o+ow}{in} \PY{n+nb}{range}\PY{p}{(}\PY{n}{n\PYZus{}runs}\PY{p}{)}\PY{p}{:}
             \PY{n}{rand\PYZus{}array} \PY{o}{=} \PY{p}{[}\PY{l+m+mi}{1000}\PY{o}{*}\PY{n}{random}\PY{p}{(}\PY{p}{)} \PY{k}{for} \PY{n}{p} \PY{o+ow}{in} \PY{n+nb}{range}\PY{p}{(}\PY{l+m+mi}{0}\PY{p}{,} \PY{n}{n}\PY{p}{)}\PY{p}{]}  \PY{c+c1}{\PYZsh{} Generate random array of size N}
             \PY{n}{inversions} \PY{o}{=} \PY{n}{inv\PYZus{}c}\PY{o}{.}\PY{n}{count}\PY{p}{(}\PY{n}{rand\PYZus{}array}\PY{p}{,}\PY{n}{ratio\PYZus{}out}\PY{o}{=}\PY{k+kc}{True}\PY{p}{)}
             \PY{n}{\PYZus{}}\PY{p}{,}\PY{n}{merge\PYZus{}sort\PYZus{}ops} \PY{o}{=} \PY{n}{merge\PYZus{}sort}\PY{o}{.}\PY{n}{sort}\PY{p}{(}\PY{n}{rand\PYZus{}array}\PY{p}{,}\PY{n}{out\PYZus{}ops} \PY{o}{=} \PY{k+kc}{True}\PY{p}{)}  \PY{c+c1}{\PYZsh{} Compute merge\PYZhy{}sort operations}
             \PY{n}{\PYZus{}}\PY{p}{,}\PY{n}{select\PYZus{}ops} \PY{o}{=} \PY{n}{selection}\PY{o}{.}\PY{n}{select}\PY{p}{(}\PY{n}{rand\PYZus{}array}\PY{p}{,}\PY{n}{out\PYZus{}ops} \PY{o}{=} \PY{k+kc}{True}\PY{p}{)}  \PY{c+c1}{\PYZsh{} Compute select operations}
             
             \PY{n}{all\PYZus{}inversions}\PY{o}{.}\PY{n}{append}\PY{p}{(}\PY{n}{inversions}\PY{p}{)}
             \PY{n}{all\PYZus{}merge\PYZus{}sort\PYZus{}ops}\PY{o}{.}\PY{n}{append}\PY{p}{(}\PY{n}{merge\PYZus{}sort\PYZus{}ops}\PY{p}{)}
             \PY{n}{all\PYZus{}select\PYZus{}ops}\PY{o}{.}\PY{n}{append}\PY{p}{(}\PY{n}{select\PYZus{}ops}\PY{p}{)}
          
\end{Verbatim}


    \begin{Verbatim}[commandchars=\\\{\}]
{\color{incolor}In [{\color{incolor}12}]:} \PY{n}{plt}\PY{o}{.}\PY{n}{plot}\PY{p}{(}\PY{n}{all\PYZus{}inversions}\PY{p}{,}\PY{n}{all\PYZus{}merge\PYZus{}sort\PYZus{}ops}\PY{p}{,}\PY{l+s+s1}{\PYZsq{}}\PY{l+s+s1}{*}\PY{l+s+s1}{\PYZsq{}}\PY{p}{)}\PY{p}{;}
         \PY{n}{plt}\PY{o}{.}\PY{n}{plot}\PY{p}{(}\PY{n}{all\PYZus{}inversions}\PY{p}{,}\PY{n}{all\PYZus{}select\PYZus{}ops}\PY{p}{,}\PY{l+s+s1}{\PYZsq{}}\PY{l+s+s1}{*}\PY{l+s+s1}{\PYZsq{}}\PY{p}{)}\PY{p}{;}
         \PY{n}{plt}\PY{o}{.}\PY{n}{legend}\PY{p}{(}\PY{p}{[}\PY{l+s+s1}{\PYZsq{}}\PY{l+s+s1}{Merge Sort}\PY{l+s+s1}{\PYZsq{}}\PY{p}{,}\PY{l+s+s1}{\PYZsq{}}\PY{l+s+s1}{Select}\PY{l+s+s1}{\PYZsq{}}\PY{p}{]}\PY{p}{)}\PY{p}{;}
         \PY{n}{plt}\PY{o}{.}\PY{n}{xlabel}\PY{p}{(}\PY{l+s+s1}{\PYZsq{}}\PY{l+s+s1}{Inversion Ratio}\PY{l+s+s1}{\PYZsq{}}\PY{p}{)}\PY{p}{;}
         \PY{n}{plt}\PY{o}{.}\PY{n}{ylabel}\PY{p}{(}\PY{l+s+s1}{\PYZsq{}}\PY{l+s+s1}{Operations}\PY{l+s+s1}{\PYZsq{}}\PY{p}{)}\PY{p}{;}
         \PY{n}{plt}\PY{o}{.}\PY{n}{title}\PY{p}{(}\PY{l+s+s1}{\PYZsq{}}\PY{l+s+s1}{Operations vs. Inversions (Experiment 1)}\PY{l+s+s1}{\PYZsq{}}\PY{p}{)}\PY{p}{;}
         \PY{n}{plt}\PY{o}{.}\PY{n}{xlim}\PY{p}{(}\PY{p}{[}\PY{l+m+mi}{0}\PY{p}{,}\PY{l+m+mi}{1}\PY{p}{]}\PY{p}{)}\PY{p}{;}
\end{Verbatim}


    \begin{center}
    \adjustimage{max size={0.9\linewidth}{0.9\paperheight}}{output_22_0.png}
    \end{center}
    { \hspace*{\fill} \\}
    
    It looks like the number of operations is constant regardless of the
number of inversions, however, a closer inspection shows that the
inversion ratio is almost always between 0.48 to 0.52 for a randomly
generated vector - a really small range. That means we'll have to force
the number of inversions to be outside this range.

To do this, we'll concatonate vectors whos values are randomly generated
within different ranges. This will let us force our arrays to be within
some inversion range. Specifically, we generate 10 subset arrays, where
each array exists in some range \([1000i+1000]\), where i is the index
of the subset.

    \begin{Verbatim}[commandchars=\\\{\}]
{\color{incolor}In [{\color{incolor}13}]:} \PY{n}{n} \PY{o}{=} \PY{l+m+mi}{20}
         \PY{n}{n\PYZus{}runs} \PY{o}{=} \PY{l+m+mi}{2000}
         \PY{n}{n\PYZus{}subsets} \PY{o}{=} \PY{l+m+mi}{10}
         \PY{n}{all\PYZus{}inversions} \PY{o}{=} \PY{p}{[}\PY{p}{]}
         \PY{n}{all\PYZus{}merge\PYZus{}sort\PYZus{}ops} \PY{o}{=} \PY{p}{[}\PY{p}{]}
         \PY{n}{all\PYZus{}select\PYZus{}ops} \PY{o}{=} \PY{p}{[}\PY{p}{]}
         
         \PY{k}{for} \PY{n}{run} \PY{o+ow}{in} \PY{n+nb}{range}\PY{p}{(}\PY{n}{n\PYZus{}runs}\PY{p}{)}\PY{p}{:}
             \PY{n}{rand\PYZus{}array} \PY{o}{=} \PY{p}{[}\PY{p}{]}
             \PY{n}{subset} \PY{o}{=} \PY{p}{[}\PY{n}{i} \PY{k}{for} \PY{n}{i} \PY{o+ow}{in} \PY{n+nb}{range}\PY{p}{(}\PY{n}{n\PYZus{}subsets}\PY{p}{)}\PY{p}{]}
             \PY{n}{shuffle}\PY{p}{(}\PY{n}{subset}\PY{p}{)}
             \PY{k}{for} \PY{n}{i\PYZus{}subset} \PY{o+ow}{in} \PY{n}{subset}\PY{p}{:}
                 \PY{n}{c\PYZus{}rand\PYZus{}array} \PY{o}{=} \PY{p}{[}\PY{l+m+mi}{1000}\PY{o}{*}\PY{n}{i\PYZus{}subset}\PY{o}{+}\PY{l+m+mi}{1000}\PY{o}{*}\PY{n}{random}\PY{p}{(}\PY{p}{)} \PY{k}{for} \PY{n}{p} \PY{o+ow}{in} \PY{n+nb}{range}\PY{p}{(}\PY{l+m+mi}{0}\PY{p}{,} \PY{n}{n}\PY{p}{)}\PY{p}{]}
                 \PY{n}{rand\PYZus{}array} \PY{o}{=} \PY{n}{rand\PYZus{}array}\PY{o}{+}\PY{n}{c\PYZus{}rand\PYZus{}array}
         
             \PY{n}{inversions} \PY{o}{=} \PY{n}{inv\PYZus{}c}\PY{o}{.}\PY{n}{count}\PY{p}{(}\PY{n}{rand\PYZus{}array}\PY{p}{,}\PY{n}{ratio\PYZus{}out}\PY{o}{=}\PY{k+kc}{True}\PY{p}{)}
             \PY{n}{\PYZus{}}\PY{p}{,}\PY{n}{merge\PYZus{}sort\PYZus{}ops} \PY{o}{=} \PY{n}{merge\PYZus{}sort}\PY{o}{.}\PY{n}{sort}\PY{p}{(}\PY{n}{rand\PYZus{}array}\PY{p}{,}\PY{n}{out\PYZus{}ops} \PY{o}{=} \PY{k+kc}{True}\PY{p}{)}  \PY{c+c1}{\PYZsh{} Compute merge\PYZhy{}sort operations}
             \PY{n}{\PYZus{}}\PY{p}{,}\PY{n}{select\PYZus{}ops} \PY{o}{=} \PY{n}{selection}\PY{o}{.}\PY{n}{select}\PY{p}{(}\PY{n}{rand\PYZus{}array}\PY{p}{,}\PY{n}{out\PYZus{}ops} \PY{o}{=} \PY{k+kc}{True}\PY{p}{)}  \PY{c+c1}{\PYZsh{} Compute select operations}
             
             \PY{n}{all\PYZus{}inversions}\PY{o}{.}\PY{n}{append}\PY{p}{(}\PY{n}{inversions}\PY{p}{)}
             \PY{n}{all\PYZus{}merge\PYZus{}sort\PYZus{}ops}\PY{o}{.}\PY{n}{append}\PY{p}{(}\PY{n}{merge\PYZus{}sort\PYZus{}ops}\PY{p}{)}
             \PY{n}{all\PYZus{}select\PYZus{}ops}\PY{o}{.}\PY{n}{append}\PY{p}{(}\PY{n}{select\PYZus{}ops}\PY{p}{)}
\end{Verbatim}


    \begin{Verbatim}[commandchars=\\\{\}]
{\color{incolor}In [{\color{incolor}14}]:} \PY{n}{plt}\PY{o}{.}\PY{n}{subplots}\PY{p}{(}\PY{l+m+mi}{1}\PY{p}{,} \PY{l+m+mi}{2}\PY{p}{,} \PY{n}{figsize}\PY{o}{=}\PY{p}{(}\PY{l+m+mi}{20}\PY{p}{,} \PY{l+m+mi}{4}\PY{p}{)}\PY{p}{)}
         \PY{n}{plt}\PY{o}{.}\PY{n}{subplot}\PY{p}{(}\PY{l+m+mi}{1}\PY{p}{,}\PY{l+m+mi}{2}\PY{p}{,}\PY{l+m+mi}{1}\PY{p}{)}
         \PY{n}{plt}\PY{o}{.}\PY{n}{plot}\PY{p}{(}\PY{n}{all\PYZus{}inversions}\PY{p}{,}\PY{n}{all\PYZus{}merge\PYZus{}sort\PYZus{}ops}\PY{p}{,}\PY{l+s+s1}{\PYZsq{}}\PY{l+s+s1}{*}\PY{l+s+s1}{\PYZsq{}}\PY{p}{)}\PY{p}{;}
         \PY{n}{plt}\PY{o}{.}\PY{n}{plot}\PY{p}{(}\PY{n}{all\PYZus{}inversions}\PY{p}{,}\PY{n}{all\PYZus{}select\PYZus{}ops}\PY{p}{,}\PY{l+s+s1}{\PYZsq{}}\PY{l+s+s1}{*}\PY{l+s+s1}{\PYZsq{}}\PY{p}{)}\PY{p}{;}
         \PY{n}{plt}\PY{o}{.}\PY{n}{legend}\PY{p}{(}\PY{p}{[}\PY{l+s+s1}{\PYZsq{}}\PY{l+s+s1}{Merge Sort}\PY{l+s+s1}{\PYZsq{}}\PY{p}{,}\PY{l+s+s1}{\PYZsq{}}\PY{l+s+s1}{Select}\PY{l+s+s1}{\PYZsq{}}\PY{p}{]}\PY{p}{)}\PY{p}{;}
         \PY{n}{plt}\PY{o}{.}\PY{n}{xlabel}\PY{p}{(}\PY{l+s+s1}{\PYZsq{}}\PY{l+s+s1}{Inversion Ratio}\PY{l+s+s1}{\PYZsq{}}\PY{p}{)}\PY{p}{;}
         \PY{n}{plt}\PY{o}{.}\PY{n}{ylabel}\PY{p}{(}\PY{l+s+s1}{\PYZsq{}}\PY{l+s+s1}{Operations}\PY{l+s+s1}{\PYZsq{}}\PY{p}{)}\PY{p}{;}
         \PY{n}{plt}\PY{o}{.}\PY{n}{title}\PY{p}{(}\PY{l+s+s1}{\PYZsq{}}\PY{l+s+s1}{Operations vs. Inversions (Experiment 2)}\PY{l+s+s1}{\PYZsq{}}\PY{p}{)}\PY{p}{;}
         \PY{n}{plt}\PY{o}{.}\PY{n}{xlim}\PY{p}{(}\PY{p}{[}\PY{l+m+mi}{0}\PY{p}{,}\PY{l+m+mi}{1}\PY{p}{]}\PY{p}{)}\PY{p}{;}
         
         \PY{n}{plt}\PY{o}{.}\PY{n}{subplot}\PY{p}{(}\PY{l+m+mi}{1}\PY{p}{,}\PY{l+m+mi}{2}\PY{p}{,}\PY{l+m+mi}{2}\PY{p}{)}\PY{p}{;}
         \PY{n}{plt}\PY{o}{.}\PY{n}{plot}\PY{p}{(}\PY{n}{all\PYZus{}inversions}\PY{p}{,}\PY{n}{all\PYZus{}merge\PYZus{}sort\PYZus{}ops}\PY{p}{,}\PY{l+s+s1}{\PYZsq{}}\PY{l+s+s1}{*}\PY{l+s+s1}{\PYZsq{}}\PY{p}{)}\PY{p}{;}
         \PY{n}{plt}\PY{o}{.}\PY{n}{plot}\PY{p}{(}\PY{n}{all\PYZus{}inversions}\PY{p}{,}\PY{n}{all\PYZus{}select\PYZus{}ops}\PY{p}{,}\PY{l+s+s1}{\PYZsq{}}\PY{l+s+s1}{*}\PY{l+s+s1}{\PYZsq{}}\PY{p}{)}\PY{p}{;}
         \PY{n}{plt}\PY{o}{.}\PY{n}{legend}\PY{p}{(}\PY{p}{[}\PY{l+s+s1}{\PYZsq{}}\PY{l+s+s1}{Merge Sort}\PY{l+s+s1}{\PYZsq{}}\PY{p}{,}\PY{l+s+s1}{\PYZsq{}}\PY{l+s+s1}{Select}\PY{l+s+s1}{\PYZsq{}}\PY{p}{]}\PY{p}{)}\PY{p}{;}
         \PY{n}{plt}\PY{o}{.}\PY{n}{xlabel}\PY{p}{(}\PY{l+s+s1}{\PYZsq{}}\PY{l+s+s1}{Inversion Ratio}\PY{l+s+s1}{\PYZsq{}}\PY{p}{)}\PY{p}{;}
         \PY{n}{plt}\PY{o}{.}\PY{n}{ylabel}\PY{p}{(}\PY{l+s+s1}{\PYZsq{}}\PY{l+s+s1}{Operations}\PY{l+s+s1}{\PYZsq{}}\PY{p}{)}\PY{p}{;}
         \PY{n}{plt}\PY{o}{.}\PY{n}{title}\PY{p}{(}\PY{l+s+s1}{\PYZsq{}}\PY{l+s+s1}{Zoomed Operations vs Inversions (Experiment 2)}\PY{l+s+s1}{\PYZsq{}}\PY{p}{)}\PY{p}{;}
         
         \PY{n}{plt}\PY{o}{.}\PY{n}{xlim}\PY{p}{(}\PY{p}{[}\PY{l+m+mi}{0}\PY{p}{,}\PY{l+m+mi}{1}\PY{p}{]}\PY{p}{)}\PY{p}{;}
         \PY{n}{plt}\PY{o}{.}\PY{n}{ylim}\PY{p}{(}\PY{p}{[}\PY{l+m+mi}{500}\PY{p}{,}\PY{l+m+mi}{1250}\PY{p}{]}\PY{p}{)}\PY{p}{;}
\end{Verbatim}


    \begin{center}
    \adjustimage{max size={0.9\linewidth}{0.9\paperheight}}{output_25_0.png}
    \end{center}
    { \hspace*{\fill} \\}
    
    \subsection{Discussion}\label{discussion}

Though we still did not end up with the full range of inversion ratios,
we had a majority of the space (0.2 to 0.8), and saw some interesting
trends. Though the average number of operations is lower for merge-sort
than select across most of inversion ratio range, the select algorithm
has more high outliers as the number of inversions grows; the select
algorithm also has much higher variance than the merge-sort algorithm.
In fact, all medians computed with the select algorithm for arrays with
an inversion ratio higher than 0.8 require more operations than the
merge-sort algorithm. Therefore, we can't guarantee that select will
actually be faster than merge-sort (especially if the inversion ratio is
high).

Interestingly, the number of operations for merge-sort is maximized when
the inversion ratio is 0.5 - the function is hyperbolic; the number of
operations decreases as the inversion ratio increases when the inversion
ratio is higher than 0.5, and the number of operations also decreases as
the inversion ratio decreases when the inversion ratio is lower than
0.5. The average number of operatiosn for select appears to be fairly
constant until the inversion ratio is high
(\textasciitilde{}\textgreater{}0.6).

The results from this programming assignment shows that in general, the
median select algorithm requires fewer operations to compute the median
than the merge-sort as the complexity is \(N\) of select and \(Nlog(N)\)
for merge-sort. However, the select algorithm isn't always less
expensive to compute the lower median. If we know that the inversion
ratio is high, we will want to use the merge-sort to find the lower
median instead of the select; furthemore, if we want our lower median
computation to take a similar amount of time in each iteration, we may
prefer the merge-sort over select algorithm.


    % Add a bibliography block to the postdoc
    
    
    
    \end{document}
